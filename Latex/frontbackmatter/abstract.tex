%*******************************************************
% Abstract
%*******************************************************
%\renewcommand{\abstractname}{Abstract}
\pdfbookmark[1]{Abstract}{Abstract}
\begingroup
\let\clearpage\relax
\let\cleardoublepage\relax
\let\cleardoublepage\relax

\chapter*{Abstract}

Adaptive photovoltaic envelopes have the potential to improve building energy performance by controlling natural lighting and solar heat gains, while simultaneously generating electricity to match the building demand. This study analyses the potential of adaptive photovoltaic envelopes, from the design process to its evaluation. The study first presents a performative design environment that combines multiple technological branches into a single automated workflow. The environment designs the form of the facade, provides feedback on its structural strength, analyses the energetic performance of the interior space, conducts a daylighting analysis, renders images and produces fabrication plans for a rapid design process. Through this methodology an adaptive photovoltaic envelope known as the Adaptive Solar Facade was successfully designed and constructed. The control strategy for the adaptive envelope is built using a model predictive control algorithm which runs successive simulations of the electricity production and building energy demand for all possible angle configurations of the panels. Angle configurations that result in minimum building energy demand, sufficient daylighting, and maximum electricity production are chosen. This control strategy can also be numerically evaluated over a year and results show that an adaptive system can have a 20\% - 80\% energy saving over an equivalent static system. The large range is due to the sensitivity of the building type. Running this numerical evaluation over eleven building use types spanning six construction periods show that modern offices, schools, and retail stores are ideal application cases. The adaptive envelopes ultimately perform best when there is a mix of both heating and cooling demands in the building space. A CO$_2$ life cycle analysis of the adaptive solar facade shows that it is favourably competitive to traditional building integrated photovoltaic systems.

\endgroup

\cleardoublepage%

\begingroup
\let\clearpage\relax
\let\cleardoublepage\relax
\let\cleardoublepage\relax

\begin{otherlanguage}{ngerman}
\pdfbookmark[1]{Zusammenfassung}{Zusammenfassung}
\chapter*{Zusammenfassung}

Adaptive Geb{\"a}udeh{\"u}llen mit integrierter Photovoltaik haben das Potential den Energiehaushalt eines Geb{\"a}udes positiv zu beeinflussen, indem die nat{\"u}rliche Lichtf{\"u}hrung und solaren W{\"a}rmegewinne gesteuert werden, w{\"a}hrend gleichzeitig Strom bedarfsgerecht produziert wird.
In dieser Arbeit wird das Potential von adaptiven Geb{\"a}udeh{\"u}llen mit integrierter Photovoltaik vom Entwurfsprozess bis zur Evaluation analysiert.
Zuerst wird eine Entwicklungsumgebung pr{\"a}sentiert, welche verschiedene technische Disziplinen zusammenf{\"u}hrt um einen automatisierten Arbeitsablauf zu erzielen.
Die Entwicklungsumgebung entwirft die Form der Fassade, gibt Feedback bez{\"u}glich der strukturellen Festigkeit, analysiert die Energieprofil der Innenr{\"a}ume, f{\"u}hrt eine Tageslichtanalyse durch, rendert Bilder und produziert Fertigungspl{\"a}ne f{\"u}r einen schnellen Entwurfsprozess.
Mit dieser Methodik konnte eine adaptive Fassade mit integrierter Photovoltaik, auch bekannt als Adaptive Solare Fassade, erfolgreich entworfen und fabriziert werden.
Zur Regelung der adaptiven Geb{\"a}udeh{\"u}lle wird ein modellgest{\"u}tzter pr{\"a}diktiver Regelalgorithmus benutzt, welcher die Stromproduktion und den Energiebedarfs f{\"u}r alle m{\"o}glichen Konfigurationen der Anstellwinkel der Panele simuliert.
Die Konfiguration der Anstellwinkel, welche den Energiebedarf des Geb{\"a}udes minimiert, gen{\"u}gend Tageslicht zul{\"a}sst und die Stromproduktion maximiert, wird gew{\"a}hlt.
Diese Regelstrategie kann auch rechnerisch {\"u}ber ein ganzes Jahr ausgewertet werden. Resultate zeigen, dass ein adaptives System 20\% - 80\% Energieeinsparungen erbringen kann im Vergleich mit einem {\"a}quivalenten statischen System.
Die grosse Spannweite der Energieeinsparungen ist auf die Sensitivit{\"a}t bez{\"u}glich der Geb{\"a}udetypen zur{\"u}ckzuf{\"u}hren.
Die Anwendung dieser Evaluation auf elf verschiedene Geb{\"a}udenutzungstypen aus sechs verschiedenen Bauperioden zeigt, dass moderne Verwaltungsgeb{\"a}ude, Schulen und Verkaufsgeb{\"a}ude ideale Anwendungsszenarien sind.
Adaptive Geb{\"a}udeh{\"u}llen funktionieren am besten, wenn ein Mix aus Heiz- und K{\"u}hlbedarf der Geb{\"a}udefl{\"a}chen besteht.
Eine Lebenszyklusanalyse der adaptiven solaren Fassade zeigt, dass diese konkurrenzf{\"a}hig zu herk{\"o}mmlichen integrierten Photovoltaiksystemen ist.

\end{otherlanguage}

\endgroup

\vfill