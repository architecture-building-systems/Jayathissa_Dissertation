% !TEX root = B99_main.tex

The results show the advantages of an adaptive system to a static system with a 20\% - 80\% net energy saving potential. These results, however, are sensitive to the building system. Decreasing the efficiency of the heating, cooling or lighting systems will give higher preference for configurations optimised for building thermal management through adaptive shading over photovoltaic electricity generation. On the contrary, a highly efficient building system will show preference of the ASF to maximise PV electricity generation by following the sun. It is therefore important to run this framework for each individual building project to evaluate the cost-benefit ratio.

The results are also sensitive to the building location. Buildings in warmer climates, such as Miami have a large cooling demand and negligible heating demands. In such cases, from an energetic perspective, there would be a limited need for adaptivity. A static system comprising of semi transparent BIPV glass, or optimally tilted louvres would be a preferred solution. From the perspective of the user however, the ability to open or close the facade shading may be more favourable than having a fixed static system which blocks the views. The influence of occupant interaction has not been included in this study and will be evaluated through measurements of the constructed prototypes.

One bottleneck in this framework is the time required to compute the optimum set of angles. A computation of 49 configurations takes approximately 20 minutes, mainly due to the slow computational speed of the radiation analysis with shading. As a result, the angle resolution was limited to 15$^{\circ}$. The radiation simulation could be accelerated by using parallel projection methods for the shading pattern calculation instead of having a discrete fine grid mesh on the PV modules. The overall computation speed could also be improved by using genetic algorithms over an exhaustive search. Through the use of these methods, a finer range of angles could be computed. 

One limitation in the model is the daylighting model. The total flux method is chosen due to computational speed. However this does not apply for large open plan office spaces or halls. Furthermore, visual comfort issues such as glare can not be modelled with this method. Ray tracing methods, such as those utilised in the software package Radiance, can overcome this, but are too computationally intensive for this analysis. A custom radiation simulation using parallel projection methods for shading, as described earlier, may allow us to quickly evaluate the glare, and minimise its effect. The lighting is also currently modelled as a closed loop on-off system. By combining the system with variable lighting may lead to more efficient and interesting daylighting strategies. 

% One limitation in the current design is the small distance between adjacent PV modules. The high density results in large self shading which, in turn reduce the overall photovoltaic efficiency of the system. Interestingly, the ASF compensates for this 

One energetic loss in the system is due to self-shading between modules. When the model was run, purely to optimise photovoltaic electricity production, the optimal angles do not follow a classic solar tracking model. Rather it follows a model similar to a two axis back tracking system \cite{lorenzo2011tracking}. This is because a solar tracking model results in high module self-shading which reduces the overall efficiency of the PV panels. If the maximisation of PV electricity generation is the primary objective for the control of PV modules, a numerical approach based on a parametric 3D model as presented in this work, may be very effective to find solutions for complex tracking system geometries. The losses could be minimised in the design stage by increasing the spacing between PV modules.

Aside from the energy savings, the coupling of the PV electricity production with the building energy consumption brings new interesting building control strategies. For example, on a sunny winters day the facade can exist in an energy optimisation position to maximise the PV electricity yield while still enabling enough solar infiltration to keep the heating and lighting demands low. When the batteries are full, the facade can switch its optimisation strategy to positions that maximise the solar infiltration to the building, thus raising the thermal mass and reducing the evening heating demands. 

