% !TEX root = B99_main.tex

This chapter presents a framework to model the energy performance of an adaptive photovoltaic envelope. This is achieved through the use of Radiance for the radiation simulations, an electrical circuit simulation for the PV production taking into account effects of self-shading, and a resistor-capacitor model for the building simulation. An exhaustive search optimisation algorithm computes the most energy efficient system configuration for control by minimising the heating, cooling and lighting load, while simultaneously maximising the PV electricity generation. This framework can be applied to evaluate different PV system geometries, building systems, building typologies and climates.

The evaluation of the Adaptive Solar Facade details the advantages of the adaptive system to a static system. The ASF is able to orientate itself to the most energy efficient position, thus finding the optimum balance between PV generation, and daylight control to minimise heating, cooling and lighting demands. When optimising for heating and lighting minimisation the ASF orients to open altitude positions at 90$^{\circ}$ to the vertical plane. When optimising for PV and cooling, the ASF selects closed position, between 15$^{\circ}$ - 45$^{\circ}$ to the vertical plane. When combining all objectives for net energy minimisation there is a tendency for the ASF to follow an optimal PV production pattern with the exception of winter evenings where it is more favourable to utilise the solar heat gains for space heating and lighting. This result however, is only restricted to this case study. A less efficient heating system, for example, will result in the ASF existing in more open configurations to utilise the solar heat gains.


The results report a 20\% - 80\% net energy saving compared to an equivalent static PV shading system depending on the efficiency of the building system. On a typical sunny winters day, the PV generation of the ASF can compensate for 62\% of the energy demand, whereas on a sunny summers day, this rises to 270\%. Over the course of the year, including cloudy days, the PV supply compensates for 61\% of the annual energy demand. This can reach 95\% in the case of a very efficient heating and cooling system with an average COP of six.

This work ultimately presents a framework for the planning and optimisation of sophisticated adaptive BIPV systems. Next stages of the research involve the utilisation of this model in the physical prototypes. The main difference in the physical model is the use of sensor data for the temperature and radiation values, as opposed to using a historical weather file. Furthermore, measuring the indoor temperature, PV electricity production, and lighting quality will create a closed loop feedback for model calibration and machine learning. 






