% !TEX root = main.tex

An adaptive solar facade, purely as a solar tracking and electricity generation technology is inferior to simple flat mounted static solutions in terms of life cycle emissions.  Classic static facade mounted Poly-Si and CIGS solutions perform 40\% to 50\% better than the ASF respectively. This is due to the additional greenhouse gas emissions, caused by the material required for the control system, supporting structure, actuators, and the energy required for actuation. A static ASF where the solar panels are orientated for optimal harvest also has a lower life cycle performance compared to classic facade systems. This is because the added structure required for optimal orientation is not compensated by the added gains in photovoltaic production.\\

However when we also consider the multi functionality of the ASF and account for energy savings to the building through adaptive shading, we have a system that yields a negative emission factor of  -906 gCO$_{2}$-eq/kWh. This is because the savings to the building system in terms of heating, cooling, and lighting offsets the embodied GWP four-fold. This demonstrates the advantage of using the PV material, not only as an electricity generation unit, but also as a building material for adaptive shading systems. In this analysis we also present a static ASF where all panels were orientated at an optimal angle of 45$^{\circ}$ to the horizontal axis. Although this solution performs well, it sacrifices user comfort. The users can not open the facade to suit their desires.\\



GWP savings through adaptive shading however are sensitive to the GWP of the electricity mix. A country with a low GWP electricity mix will result in lower operational GWP savings than a country with a high GWP electricity mix. For example, an ASF installed in Switzerland has a higher emission factor of 53.5 gCO$_{2}$-eq/kWh.\\

Although it is favorable to install an ASF in Germany, it still has benefits in countries such as Switzerland. For instance, with an emission factor 53\% less than the standard mix, it contributes to a nuclear free energy mix. Furthermore, it provides interesting design options for architects where they can install PV in locations which were previously not possible. Thus increasing BIPV potential.\\

When designing an ASF architects and engineers may consider: 
\begin{itemize}
\item The added benefit of a highly adaptable shading element
\item The trade-off between soft robotic actuators and servo motors for actuation. Although the investigated soft robotic actuator has an embodied GWP three times lower than a servo motor, it requires three times more energy to actuate. Purely from an LCA perspective, if more than 6 actuations are required a day, servo motors would be the preferred solution. 
\item Control system electronics cost 27.5kgCO$_{2}$-eq/kg and play a large contribution in human toxicity, freshwater eutriphication and terrestrial acidification. They should therefore be carefully designed. However increasing the resolution of the ASF control system to allow each panel to be independently actuated only increases the emission factor by 1.6gCO$_{2}$-eq/kWh.
\item The structural support system in our current analysis used a stainless steel frame representing 22\% of our total embodied carbon emissions, 21\% of terrestrial acidification, 44\% of metal depletion, and 25\% of photochemical oxidant formation. Redesigning the frame to use less stainless steel, or an alternative material with a lower life cycle impact, such as plain steel, should be considered.
\item If the ASF is installed in front of an opaque building surface then the advantages of adaptive shading are not present. In this case, a static, flat mounted system is a preferred design choice. 
\end{itemize}
 
One limitation of the LCA is that the analysis focuses on a single office room. Expanding the analysis to the entire building, or urban level may yield different results. The LCA also assumes that the user will not override the system. In practice the facade will adapt to the desires of the user. 

The LCA also excludes other aspects of building system such as the downsizing of heating and cooling appliances, the use of DC electricity on-site, and the increase in user comfort.

