% !TEX root = main.tex


As an electricity producing device the ASF is outperformed by fixed PV systems. However it comes with the added benefit of building integration and multi functionality, i.e. allowing for better control of solar loads and user comfort. When adding these aspects to the comparison (i.e. acounting for HVAC savings), the system becomes favourably competitive to a traditional PV system as it has a negative emission factor of -906gCO${_2}$/kWh. These advantages however, will not be present if the ASF is installed over an opaque building surface. It is therefore preferable to install static systems over opaque facades, and keep the adaptive system for glazed facades only.\\

The design of an ASF can greatly influence the results. Varying factors such as the choice of actuators, the complexity of the control system, and the structural support can change the emission factor. The largest variable however is the emission factor of the grid electricity mix. The building operational savings in heating, cooling, and lighting will have a CO${_2}$ saving based on the grid electricity mix. 


Future research will validate the assumptions to building energy consumption through experimentation, and test the users response. This will be conducted on the ETH House of Natural Resources living lab where an example of an ASF has already been constructed \cite{nagy2016adaptive}. Further numerical simulations of the ASF on different building typologies, building systems and climates will enable us to specifically target the best application scenario. \\

To conclude, I demonstrate that BIPV systems and adaptive shading elements complement each other successfully. One can see an improvement in environmental performance of the PV technology, and create new architectural possibilities for the aesthetic integration of PV panels over glazed building surfaces, thus expanding BIPV potential. 

