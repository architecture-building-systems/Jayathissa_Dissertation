% !TEX root = 99_main.tex

In this paper, we have evaluated the energetic performance of the adaptive solar facade (ASF) over 11 building archetypes spanning six construction periods. We notice that the ASF performs most efficiently in an environment dominated by cooling demands as the angles of photovoltaic panels to reduce the cooling demand also maximise the photovoltaic electricity supply. When compared to a plain glazed facade with no shading system, we observe large energy saving reductions in buildings that have high cooling demands and low heating demands. When compared to an equivalent static photovoltaic shading system, the inverse is true: larger energy saving potentials are present in buildings with low cooling demands and high heating demands. This is because the static system is not capable of opening its panels during times where heating is required, thus disadvantaging buildings with a large heat demand. 

For shading system designers, there is an important balance between the building archetype and the type of shading system to be used. For buildings with poorly performing envelopes, maximising solar heat gains is important to reduce heating loads. A window without shading, or a system with manually controlled Venetian blinds may be the optimal solution. Modern buildings with good insulation and large heat gains have a high cooling demand. In these cases, the ASF performs very well and would be a recommended solution. However, when the cooling demands are very high, such as in a modern gym, it would be more cost effective to install a simpler, static photovoltaic shading system at an optimum solar angle. 

Further expansion of this research should be conducted to evaluate the impact of the building system. Increasing the efficiency of the heating, cooling, or lighting system will give higher preference for configurations optimised for photovoltaic electricity generation over building thermal management. 

The analysis ultimately helped to clearly identify possible application cases of an adaptive solar facade with the highest benefit in terms of energy efficiency. The methodology proposed may also be useful for the identification of suitable application cases for other shading system types. 

%Even though the ASF is capable of reducing the overall net energy consumption, it is not sufficient to cover the added costs and complexity. A non shaded window, or a system with a manually controlled fabric shading or venetian blinds is therefore the optimal solution. For modern buildings with good insulation and large heat gains, there is a sufficient cooling demand. In these cases, the ASF performs very well and would be a recommended solution. However when the cooling demands are very high, such as in modern gym, it would be more cost effective to install a static photovoltaic shading system at an optimum solar angle. 


