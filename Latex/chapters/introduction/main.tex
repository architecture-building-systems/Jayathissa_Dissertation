% !TEX root = ../../thesis.tex

\chapter{Introduction}
\label{ch:introduction}

% \dictum[Immanuel Kant]{%
%   Sapere aude! Habe Mut, dich deines eigenen Verstandes zu bedienen! }%
% \vskip 1em


- The potential of Adaptive Architecture

- The CO2 demand of the built environment

- Ways in which adaptive architecture, combined with BIPV can mitigate these losses

- Previous Research 

- Introduction to the ASF to be used as a case study throughout the paper


\section{Research Questions}

The four questions addressed in this research are 

\begin{itemize}
\item How can complex architectural components, such as the ASF be designed and constructed? 
\item How can a photovoltaic envelope be controlled to be adaptive?
\item What is the energy saving potential of an adaptive photovoltaic envelope?
\item How does the energy saving potential vary for different building types?
\item What is the life cycle CO$_2$ saving potential of an adaptive photovoltaic facade?

\end{itemize}

\section{Organisation of the Thesis}

The remainder of this thesis is composed of three journal papers and one conference paper. Chapter \ref{ch:asfDesign} introduces the parametric design environment, which was created for rapid iterative development of the ASF. This chapter also introduces some of the design elements of the ASF. Chapter \ref{ch:asfSimulation} introduces the model predictive control strategy to allows for adaptive control. This chapter first introduces the simulation methodology, and then discusses the energy saving potential of an ASF system. Chapter \ref{ch:asfArchetype} takes on the model from Chapter \ref{ch:asfSimulation} and runs an evaluation on eleven different building use types spanning six construction periods. Chapter \ref{ch:asfLCA} then takes the results of the energy simulation methodology and assesses the carbon life cycle cost. Finally Chapter \ref{ch:conclusion} concludes the thesis. 