% !TEX root = ../../thesis.tex

\chapter{Introduction}
\label{ch:introduction}

\dictum[Alex Millane]{%
  I'm drinking a gin and tonic, without gin}%
\vskip 1em

A building, in its original manifestation, is a shelter. A means to protect the human body from harsh external conditions. And within this we ascribe the notion of the envelope, the barrier between the external and internal environments. It is the barrier that protects us from frigid temperatures, shades us from solar rays, and keeps us dry when a storm passes by. And over time, we have not just developed the quality of our envelopes, but also technologies that enable us to manufacture interior environments. The combination of heating, cooling, lighting and air handling enables us to exclude the energies of the exterior and form hermatic envelopes. Buildings transformed from mere shelters, to places of comfort where we now spend 87\% of our lives \cite{klepeis2001national}. We have in essence, become an indoor species. \

Unfortunately, the manufacture of interior environments comes with a large environmental impact. Buildings are currently responsible for 32\% of our final energy use and 19\% of our total greenhouse gas emissions \cite{IPCC}. There is, however, a 50\% - 90\% emission reduction potential using existing technologies \cite{IPCC}. On one hand, the efficiency of our manufactured interior environment can be increased. We can install more efficient systems to manufacture this energy at a lower environmental cost, and by increasing the isolation properties of our envelopes, we reduce the loss of this energy to the exterior. On the other hand, we can rewind the clock of architectural history and move back to a time where we did not manufacture internal environments, but rather mediated the external energies to fulfil that of the interior. These strategies, commonly described as passive design strategies, include aspects such as natural ventilation, thermal storage, and static shading. 

But instead of rewinding the clock of architectural history, can we not look ahead. Instead of passive mediation of the external environment, is there room for active mediation. A building envelope that can actively utilise the energies of the exterior and 



- The potential of Adaptive Architecture

- The CO2 demand of the built environment

- Ways in which adaptive architecture, combined with BIPV can mitigate these losses

- Previous Research 

- Introduction to the ASF to be used as a case study throughout the paper


\section{Research Questions}

The four questions addressed in this research are 

\begin{itemize}
\item How can complex architectural components, such as the ASF be designed and constructed? 
\item How can a photovoltaic envelope be controlled to be adaptive?
\item What is the energy saving potential of an adaptive photovoltaic envelope?
\item How does the energy saving potential vary for different building types?
\item What is the life cycle CO$_2$ saving potential of an adaptive photovoltaic facade?

\end{itemize}

\section{Organisation of the Thesis}

The remainder of this thesis is composed of three journal papers and one conference paper. Chapter \ref{ch:asfDesign} introduces the parametric design environment, which was created for rapid iterative development of the ASF. This chapter also introduces some of the design elements of the ASF. Chapter \ref{ch:asfSimulation} introduces the model predictive control strategy to allows for adaptive control. This chapter first introduces the simulation methodology, and then discusses the energy saving potential of an ASF system. Chapter \ref{ch:asfArchetype} takes on the model from Chapter \ref{ch:asfSimulation} and runs an evaluation on eleven different building use types spanning six construction periods. Chapter \ref{ch:asfLCA} then takes the results of the energy simulation methodology and assesses the carbon life cycle cost. Finally Chapter \ref{ch:conclusion} concludes the thesis. 