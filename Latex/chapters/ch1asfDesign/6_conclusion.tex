% !TEX root = 99_main.tex

This chapter presents a practical PDE for the design and fabrication of kinetic architectural elements. The PDE is capable of combining the multiple fields of structural engineering, energy engineering, control engineering, industrial design and architecture into one uniform environment, allowing the designers to handle the complexities of a multidisciplinary project. 

The major advantage of this environment is the considerable decrease in time between design iterations. Traditionally each of the respective stakeholders in the project would work on their individual design, and then exchange information in a meeting. With the PDE, the meeting is transformed from an information exchange session, to a design session where all stakeholders can collaboratively influence the design and immediately see the necessary results.  With traditional methods, five design iterations would normally take a month, whereas with the PDE, this can be condensed to a few hours. The design environment also develops with the project allowing for new parametric inputs, or new outputs to be created. This can be easily done with collaborative software management tools such as git, a distributed version control system.

One disadvantage is the overhead required to manage a PDE. Like a BIM manager, a PDE manager is required and must have a careful overview of the software. As the software ultimately determines the final form of the design, any errors in the software can be detrimental to the final design. It was therefore necessary for all stakeholders to conduct a final independent analysis prior to the submission of the final design. 

One limiting factor in the design of the PDE is the computational time. The full annual energetic analysis, for example, may take six hours to solve. Simplifications were therefore made to accelerate this process during the design stage, and the full complex evaluation was conducted afterwards to validate the simplified model.

It is also important to determine what aspects of the design should be contained within the PDE and what should be designed with classical methods. Essentially, all aspects where there could be conflicts between the stakeholders were included in the PDE, whereas many of the design details, such as the mounting brackets to the building, can still be designed independently and imported into the PDE as a static object. Over time, some static objects, such as the cantilevered bracket of the PV module became parametrised. 

Ultimately, this chapter presents a further step in the field of performative design by showing how a system as complex as a kinetic photovoltaic envelope, can be developed, prototyped and finally fabricated by a small team of four designers. The methodology can be utilised for any building component where multiple technological branches are required.

