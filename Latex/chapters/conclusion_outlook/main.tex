% !TEX root = ../../thesis.tex

\chapter{Conclusion}
\label{ch:mainconclusion}

\dictum[Zhuangzi]{%
   Words exist because of meaning. Once you've gotten the meaning, you can forget the words. }
\vskip 1em

This thesis was based on the hypothesis that an adaptive photovoltaic envelope can reduce the CO$_2$ footprint of the built environment. In order to test this hypothesis, I present how such a system could be designed, constructed, controlled, and finally, evaluated. \\

I first present the performative design environment for the fabrication of kinetic architectural elements. Here I show how the multiple fields of structural engineering, energy engineering, control engineering, industrial design, and architecture can be combined into a single automated work-flow that accelerates the design process. 


I then present how the adaptive system can be controlled using a model predictive algorithm which combines building energy demand and solar radiation simulations. From the simulation outputs, the adaptive system can determine the best physical configuration to minimise the net energy building demand. This framework can be used, to not only control the facade, but to evaluate the energy saving potential of an adaptive photovoltaic envelope over a static system. Results show that there is a 20\% - 80\% energy saving potential compared to an equivalent static system. 

The range of these results is large as they are heavily dependent on the building type. This framework is therefore run for 11 different building use types spanning six construction periods. Results show that the ASF performs best in environments where there is a mix of both heating and cooling demands. For buildings that predominantly have cooling demands, a simple static system at an optimum solar angle would be the most cost-effective solutions. Likewise, for a building that predominantly consists of heating demands, a window without shading, or a manually controlled Venetian blind may be optimal. The ASF performs best in modern offices, retail stores, food stores and schools. 

Finally, I extract the results of a modern office based simulation and conduct a CO$_2$ life cycle analysis. Results show that the ASF, running purely as an electricity producing device, is outperformed by static PV systems by 50\%. However, when I include the energy saving to the building interior, the ASF becomes favourably competitive to a traditional PV system. \\


This thesis is largely centred around the development of the ASF technology. However it is not the technology that is of interest, rather the proof that such a technology can be built, controlled, and reduce the CO$_2$ footprint of a building. By doing so, this thesis positions itself as a further stepping stone in the adaptive transformation of our built environment. As the multiple fields of robotics, architecture, and energy engineering continue to merge, novel adaptive concepts of will be developed. The methods presented in this thesis, can then facilitate the transformation of these concepts into physical products. \\

There are still major hurdles that must be overcome for this transition to occur. From discussions I have had with designers of the Al Bahar towers and the Arab World Institute, and my own experiences, maintenance is the largest hurdle. Most electrical and mechanical components will only have a warranty of 20 years, whereas a building must last for at least 50 years. This could be overcome with regular maintenance checks, however the costs will be borne by the building owner which would restrict the market uptake of the technology. Neglecting the maintenance of the adaptive envelope will eventually lead to failure, which results in immense occupant dissatisfaction. For example, a failure on the prototype at the House of Natural Resources prevented the panels from opening, leaving the occupants in a cold dark office for over a month. Fortunately, this was still a prototype; however equivalent failures on a commercial scale may have a detrimental impact on the future of adaptive envelopes. \\

I however believe that with our rate of technological advancement, we will see adaptive envelopes of some form enter the market. It is simply a matter of time.




\section{Outlook}
\label{ch:mainoutlook}

The next imminent steps lie in the hands of future architects. In essence, this thesis is a feasibility study that describes how an adaptive system can be designed, controlled and evaluated; using the adaptive solar facade as a case study. In reality, adaptive architecture will not just be limited to kinetic envelopes, but any technology that has the potential to vary its property. Examples could include walls with varying thermal resistances, or variable ventilation systems. 

The kinetic envelope however, has a property that other adaptive technologies lack. And that is the architectural expression of the building. An adaptive facade evolves a building from a static system to one that feels alive and changes with variation in the season, the day, the weather, or its use. 

One remaining step in this technology that hasn't been addressed in this thesis is the interaction with users. The users currently have the ability to override the adaptive algorithm in order to open or close the panels. This override, however, should also be integrated into a machine learning algorithm to enable the facade not just to adapt to the external conditions, but to the desires of the user. This research will be possible once the constructed ASF is installed on the HiLo building.


