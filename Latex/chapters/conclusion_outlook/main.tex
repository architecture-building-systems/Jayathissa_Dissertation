% !TEX root = ../../thesis.tex

\chapter{Conclusion}
\label{ch:conclusion}

% \dictum[Vincent van Gogh]{%
%    I dream my painting and I paint my dream. }
% \vskip 1em

This thesis presents a hollistic analysis of adaptive photovoltaic envelopes from design and control, to evaluation. We first present the performative design environment for the fabrication of kinetic architectural elements. Here we show how the multiple fields of structural engineering, energy engineering, control engineering, industrial design, and architecture can be combined into a single automated workflow that accelerates the design process. We then present how the adaptive system can be controlled using a model predictive algorithm which combines building energy demand and solar radiation simulations. From the simulation outputs, the adaptive system can determine the best physical configuration to minimise the net energy building demand. We can use this framework to not only control the facade, but to evaluate the energy saving potential of an adaptive photovoltaic envelope over a static system. Our results show that there is a 20\% - 80\% energy saving potential compared to an equivalent static system. The range of these results are large as they are heavily dependent on the building type. We therefore run this framework for 11 different building use types spanning six construction periods. Our results show that the ASF performs best in environments where there is a mix of both heating and cooling demands. For buildings that predominantly have cooling demands, a simple static system at an optimum solar angle would be the most cost effective solutions. Likewise for a building that predominantly consists of heating demands, a window without shading, or a manually controlled Venetian blind may be optimal. The ASF performs best in modern offices, retail stores, food stores and schools. Finally, we extract the results of a modern office based simulation and conduct a CO$_2$ life cycle analysis. Our results show that the ASF, running purely as an electricity producing device, is outperformed by static PV systems by 50\%. However when we include the energy saving to the building interior, the ASF becomes favourably competitive to a traditional PV system. 


\section{Outlook}
\label{ch:outlook}

- So what does this mean for the future of adaptive architecture?

- In essence, this thesis is a feasibility study. We have shown that with modern design tools, and in house software, it is possible to design and construct complex adaptive envelopes. We have proposed an adaptive control methodology, and shown that for certain building types there is a large net CO$_2$ saving potential. 

- Adaptive architecture, however is not limited to just kinetic solar facades. This is just one of many possible adaptive systems that could be brought to reality. Any technology that has the potential to vary its property, can in principal be reprogrammed with adaptive algorithms. Examples could include walls with varying thermal resistances, or variable ventilation systems. 

- Architectural expression of kinetic skins. A building is not longer a static system, but one that changes its form with the changes in the season, the day, the weather, or its use. 

- The one remaining step in this technology is the interaction with the users. The building users currently have the ability to overide the adaptive algorithm, however this overide should be included in a machine learning algorithm to improve the user comfort of the adpative algorithms. 

Next steps in this research involve, testing the constructed ASF's in the HiLo building and comparing the performance against the models. When complete the algorithms should also be complemented with machine learning feedback from the occupant. 

- 
